\RequirePackage{luatex85}
\PassOptionsToPackage{shorthands=off}{babel}
\makeatletter
\disable@package@load{fontenc}
\makeatother
\let\oldlooseness=\looseness
\documentclass{csbulletin}
\selectlanguage{czech}
\setcounter{secnumdepth}{3}
\usepackage{titlesec}
\titlelabel{\thetitle\enspace}
\usepackage{luavlna}
\usepackage[strict]{lua-widow-control}
\usepackage{csquotes}
\usepackage[
  backend=biber,
  style=iso-numeric,
  sortlocale=cs,
  autolang=other,
  bibencoding=UTF8,
  mincitenames=2,
  maxcitenames=2,
  isbn=false,
  shortnumeration=true,
]{biblatex}
\addbibresource{main.bib}
\usepackage[
  linesnumbered,
  ruled,
  vlined,
]{algorithm2e}
\renewcommand\algorithmcfname{Algoritmus}
\usepackage{minted}
\usemintedstyle{bw}
\setminted{firstnumber=last, linenos}
\newenvironment{mintedblock}{%
  \par\vspace{\topsep}\vspace{\partopsep}%
  \begingroup
  \fvset{listparameters=\setlength{\topsep}{0pt}\setlength{\partopsep}{0pt}}%
}{%
  \endgroup
  \par\vspace{\topsep}\vspace{\partopsep}%
}
\usepackage{multicol}
\usepackage[
  implicit=false,
  hidelinks,
]{hyperref}
\newcommand\vref[1]{\ref{#1} na straně~\pageref{#1}}

\newcommand\acro[1]{\textsc{\MakeLowercase{#1}}}
\newcommand\pkg{\textsf}

\begin{document}

\singlechars{czech}{AaIiVvOoUuSsZzKk}

\title{Sazba textu české lidové písně „Když jsem já sloužil“ pomocí modulu \pkg{l3seq} jazyka expl3}
\EnglishTitle{Typesetting Lyrics of Czech Folksong ``\foreignlanguage{czech}{Když jsem já sloužil}'' using the \pkg{l3seq} Module of Expl3 Language}
\author{Vít Starý Novotný}
\podpis{Vít Starý Novotný, witiko@mail.muni.cz}
\maketitle[1ex]
\vspace*{-1.25cm}

\begin{abstract}
Jazyk plain \TeX{} vznikl pro sazbu knih a turingovsky úplným programovacím jazykem se stal až na konci svého vývoje. Zatímco příprava textu dokumentů a úpravy vzhledu jsou v plain \TeX u přímočaré, programování naráží na chybějící základní datové struktury a na odloženou expanzi maker, která neodpovídá běžnému vyhodnocování v moderních imperativních jazycích.

Ve stroji Lua\TeX{} je možné programovat také v imperativním programovacím jazyce Lua. Jazyk Lua sice zmíněnými neduhy plain \TeX u netrpí, ale komunikace mezi \TeX em a Luou není přímočará a při předávání dat dochází ke ztrátě důležitých informací, jako jsou kategorie \TeX ových znaků.

Programovací jazyk expl3 nabízí zlatou střední cestu a umožňuje uživatelům programovat v \TeX u způsobem, na který jsou zvyklí z moderních imperativních programovacích jazyků.

V tomto článku představuji modul \pkg{l3seq} jazyka expl3, který poskytuje datovou strukturu seznamu. Možnosti modulu demonstruji na sazbě textu české lidové písně \emph{Když jsem já sloužil}.
Implementaci v jazyce expl3 porovnávám s implementací v plain \TeX u.
\end{abstract}
\vspace*{0.15cm}
\klicovaslova: \LaTeX3, expl3, \pkg{l3seq}, \pkg{xparse}, plain \TeX, Op\TeX, texty písní
\vspace*{1cm}

\TeX{} je strojový kód světa digitální sazby, který programátorům poskytuje minimum vysokoúrovňových abstrakcí. V předchozím článku~\cite[sekce~2]{novotny2022vysokourovnove} jsem představil vysokoúrovňový programovací jazyk expl3~\cite{latex2023expl3, latex2023interfaces}, díky kterému mohou uživatelé programovat v \TeX u způsobem, na který jsou zvyklí z moderních imperativních programovacích jazyků. V dalším článku jsem představil modul \pkg{l3keys} jazyka expl3~\cite[sekce~3.2]{starynovotny2023napadovnik}, který umožňuje přípravu datových sběrnic. V tomto článku představuji modul \pkg{l3seq} jazyka expl3, který poskytuje datovou strukturu seznamu. Možnosti modulu demonstruji na sazbě textu české lidové písně \emph{Když jsem já sloužil}~\cite{hroudova2015kdyz}. Implementaci v jazyce expl3 porovnávám s implementací v plain \TeX u.

Nejprve v sekci~\ref{sec:algoritmus} popisuji algoritmus pro generování textu písně. V sekci~\vref{sec:implementace} algoritmus implementuji pomocí modulu \pkg{l3seq} jazyka expl3. Následně v sekci~\vref{sec:rozhrani} vytvářím uživatelské rozhraní pomocí \LaTeX ového balíčku \pkg{xparse}. Dále v sekci~\vref{sec:sazba} ukazuji vysázený text písně. Nakonec v sekci~\vref{sec:plain} algoritmus implementuji v plain \TeX u a porovnávám obě implementace. V sekci~\vref{sec:zaver} shrnuji poznatky z článku a jejich přínos pro čtenáře.

\section{Algoritmus pro generování textu písně}
\label{sec:algoritmus}

V této sekci představuji českou lidovou píseň \emph{Když jsem já sloužil}~\cite{hroudova2015kdyz}.
Nejprve píseň zasazuji do historického kontextu a popisuji strukturu textu písně.
Následně popisuji algoritmus pro generování textu písně.

Píseň \emph{Když jsem já sloužil} je autobiografie mladého muže, který sloužil devět let jako zemědělský dělník. Vypravěč zmiňuje robotu a omezování sňatků vrchností, což děj zasazuje do dob Rakousko-Uherska před zrušením nevolnictví roku 1781.

Text písně se dělí na sloky, které popisují jednotlivé roky vypravěčovy služby. První věta každé sloky udává rok služby a obdrženou odměnu, např.: ,,Když jsem já sloužil to \emph{3. léto}, vysloužil jsem si \emph{husičku} za to.`` Následuje dlouhé souvětí s popisem odměn za dosavadní roky služby počínaje současným rokem a konče prvním rokem, např.: ,,A ta husa chodí bosa a ta kačka bláto tlačká a to kuře krákoře, běhá po dvoře.`` Sloku zakončuje vypravěč domněnkou o aktivitách své partnerky, např.: ,,Má panenka pláče doma v komoře.``

Pokud si odměny ve čtvrtém pádu (např. ,,husičku``) a popisy odměn (např. ,,ta husa chodí bosa``) uložíme do dvou seznamů \texttt{čtvrté\_pády} a \texttt{popisy} délky $N=9$, můžeme text písně vygenerovat Algoritmem~\ref{algo:algoritmus}. V algoritmu je použita funkce \textbf{zip}, která postupně navrací prvky obou seznamů jako dvojice.

\begin{algorithm}
\DontPrintSemicolon
\KwData{Seznamy \texttt{čtvrté\_pády} a \texttt{popisy} délky \(N\)}
$i\leftarrow 1$;
\texttt{popisy\_pozpátku}${}\leftarrow{}$[]\;
\For{\textup{\texttt{čtvrtý\_pád}, \texttt{popis} \textbf{in} \textbf{zip}(\texttt{čtvrté\_pády}, \texttt{popisy})}}{
    \eIf{\(i < N\)}{
        \textbf{print} \texttt{"když jsem já sloužil to "} + \(i\) + \texttt{". léto"}\;
    }{
        \textbf{print} \texttt{"když jsem já sloužil poslední léto"}
    }
    \textbf{print} \texttt{"vysloužil jsem si "} + \texttt{čtvrtý\_pád} + \texttt{" za to"}\;
    $j\leftarrow i$;
    \texttt{popisy\_pozpátku}${}\leftarrow{}$[\texttt{popis}]${}+{}$\texttt{popisy\_pozpátku}\;
    \For{\textup{\texttt{popis\_pozpátku} \textbf{in} \texttt{popisy\_pozpátku}}}{
        \leIf{\(j < N\)}{
            \textbf{print} \texttt{"a "} + \texttt{popis\_pozpátku}\;
        }{
            \textbf{print} \texttt{popis\_pozpátku}
        }
        $j\leftarrow j-1$\;
    }
    \eIf{\(i < N\)}{
        \textbf{print} \texttt{"má panenka pláče doma v komoře"}\;
    }{
        \textbf{print} \texttt{"má panenka stele postel v komoře"}
    }
    $i\leftarrow i+1$\;
}
\caption{Text písně \emph{Když jsem já sloužil}}
\label{algo:algoritmus}
\end{algorithm}

\section{Implementace algoritmu pomocí modulu \pkg{l3seq}}
\label{sec:implementace}

V této sekci rozpracujeme soubor \texttt{kdyz-jsem-ja-slouzil.sty}, který bude implementovat Algoritmus~\ref{algo:algoritmus} pomocí modulu \pkg{l3seq} jazyka expl3. Hotový soubor \texttt{kdyz-jsem-ja-slouzil.sty} můžeme stáhnout online~\cite{starynovotny2023sazba}.

\begin{mintedblock}
\inputminted{tex}{kdyz-jsem-ja-slouzil-01.sty}
\inputminted{tex}{kdyz-jsem-ja-slouzil-02.sty}
\end{mintedblock}

\section{Uživatelské rozhraní pomocí \LaTeX ového balíčku \pkg{xparse}}
\label{sec:rozhrani}

Programovací jazyk expl3 má pro uživatele \TeX u poměrně neobvyklou syntax. V této sekci ukončíme soubor \texttt{kdyz-jsem-ja-slouzil.sty} a pomocí \LaTeX ového balíčku vytvoříme uživatelské rozhraní, které bude pro uživatele přirozenější.

\begin{mintedblock}
\inputminted{tex}{kdyz-jsem-ja-slouzil-03.sty}
\inputminted{tex}{kdyz-jsem-ja-slouzil-04.sty}
\end{mintedblock}

\section{Sazba textu písně}
\label{sec:sazba}

V této sekci vytvoříme a vysázíme ukázkový soubor \texttt{priklad-latex.tex}. Soubor nejprve načte soubor \texttt{kdyz-jsem-ja-slouzil.sty}, který jsme připravili v sekcích \ref{sec:implementace} a \ref{sec:rozhrani}, a následně vysází text písně \emph{Když jsem já sloužil}.

\begin{mintedblock}
\inputminted[firstnumber=1]{tex}{example-01.tex}
\inputminted{tex}{example-02.tex}
\inputminted{tex}{example-03.tex}
\end{mintedblock}

\noindent
Soubor zpracujeme příkazem \texttt{lualatex priklad-latex} a obdržíme tento výstup:

\input kdyz-jsem-ja-slouzil-00.sty
\input kdyz-jsem-ja-slouzil-02.sty
\input kdyz-jsem-ja-slouzil-04.sty
\input kdyz-jsem-ja-slouzil-05.sty
\input example-00.tex
\begingroup
\parindent=0pt
\parskip=.5\baselineskip plus 2pt
\input example-02.tex
\par
\endgroup

\section{Implementace algoritmu pomocí plain \TeX u}
\label{sec:plain}

V této sekci vytvoříme soubor \texttt{kdyz-jsem-ja-slouzil.opm}, který bude implementovat Algoritmus~\ref{algo:algoritmus} v jazyce plain \TeX. Pro usnadnění použijeme formát Op\TeX, který přidává nad rámec plain \TeX u podporu pro jmenné prostory~\cite[sekce~2.2.3]{olsak2023optex}, které využijeme při definici lokálních proměnných, a pro vícesloupcovou sazbu~\cite[sekce~1.5.2]{olsak2023optex}, kterou využijeme v ukázkovém souboru.

\begin{mintedblock}
\inputminted[firstnumber=1]{tex}{kdyz-jsem-ja-slouzil.opm}
\end{mintedblock}

Následně vytvoříme ukázkový soubor \texttt{priklad-optex.tex}. Soubor nejprve načte soubor \texttt{kdyz-jsem-ja-slouzil.opm}, který jsme připravili v této sekci, a následně vysází text písně \emph{Když jsem já sloužil}.

\begin{mintedblock}
\inputminted[firstnumber=1]{tex}{example-optex.tex}
\end{mintedblock}

\noindent
Soubor zpracujeme příkazem \texttt{optex priklad-optex} a obdržíme výstup ze sekce~\ref{sec:sazba}.

Nyní porovnáme implementaci v jazyce expl3 ze sekcí \ref{sec:implementace} a \ref{sec:rozhrani}\, s implementací v jazyce plain \TeX{} z této sekce. Implementace v jazyce expl3 má takřka dvojnásobnou délku. Délkový rozdíl je způsobený především rozvláčností jazyka expl3, ne složitostí kódu, a dopad na čitelnost kódu je tedy sporný.

Implementace seznamů v plain \TeX u není přímočará a vyžaduje opatrnou práci s expanzí. Implementace z této sekce se proto seznamům vyhýbá a nahrazuje je hašovou tabulkou. Díky tomu je implementace v plain \TeX u stručná a čitelná, ale není obecná: Programátor se musí přizpůsobit omezením plain \TeX u a připravit kód na míru řešenému problému, zatímco jazyk expl3 disponuje bohatou knihovnou datových struktur, které je možné použít pro řešení různých problémů.

Recenzent článku navrhl ještě jednodušší řešení v idiomatickém plain \TeX u, které implementaci algoritmu nevyčleňuje do samostatného balíčku a které jednotlivé sloky okamžitě sází, aniž by si je předtím ukládalo do datové struktury. Navržené řešení s drobnými úpravami uvádím se svolením recenzenta:

\begin{mintedblock}
\begin{minted}[firstnumber=1]{tex}
\fontfam [LMFonts]
\parskip = 6pt plus 2pt
\parindent = 0pt
\def \sloka #1#2#3{\par
  \ifx^#3^\else
    \edef \popisy {%
      #3\ifx \popisy \empty \else \nl a \fi \popisy
    }%
  \fi
  Když jsem já sloužil #1 léto,\nl
  vysloužil jsem si #2 za to.\nl
  A \popisy, běhá po dvoře,\nl
  má panenka \cosidela.\par
}
\def \popisy {}
\def \cosidela {pláče doma v komoře}
\sloka {to první}  {kuřátko}   {to kuře krákoře}
\sloka {to druhé}  {kachničku} {ta kačka bláto tlačká}
\sloka {to třetí}  {husičku}   {ta husa chodí bosa}
\sloka {to čtvrté} {vepříka}   {ten vepř jako pepř}
\sloka {to páté}   {telátko}   {to tele hubou mele}
\sloka {to šesté}  {kravičku}  {ta kráva mléko dává}
\sloka {to sedmé}  {volečka}   {ten vůl jako kůl}
\sloka {to osmé}   {botičky}   {ty boty do roboty}
\def \cosidela {stele postel v komoře}
\sloka {poslední}  {děvčátko}  {}
\bye
\end{minted}
\end{mintedblock}

Toto řešení pěkně ilustruje výhody jazyka expl3, protože se odchyluje od požadovaného textu písně: Popis odměn v posledním roce začíná slovy ,,A ty boty do roboty`` namísto ,,To děvčátko jak poupátko``. Oprava této chyby není přímočará a vyžaduje přepis většiny makra \verb"\sloka":

\begin{mintedblock}
\begin{minted}[firstnumber=4]{tex}
\def \sloka #1#2#3{\par
  \edef \popisy {%
    #3\ifx \popisy \empty \else \nl a \fi \popisy
  }%
  Když jsem já sloužil #1 léto,\nl
  vysloužil jsem si #2 za to.\nl
  % V poslední sloce vynech počáteční spojku „A“
  % a kapitalizuj první písmeno popisu odměny.
  \edef \cislosloky {#1}%
  \edef \posledni   {poslední}%
  \ifx \cislosloky  \posledni
    \def \prvnivelke ##1{\uppercase {##1}}%
    \expandafter \prvnivelke \popisy
  \else
    A \popisy
  \fi, běhá po dvoře\nl
  má panenka \cosidela.\par
}
\end{minted}
\end{mintedblock}

\noindent
Poslední sloku po této úpravě vysázíme následně: \verb"\sloka{poslední}{děvčátko}"
\verb"{to děvčátko jak poupátko}".

V naší implementaci v jazyce expl3 ze sekce~\ref{sec:implementace} by pro podobnou změnu stačilo upravit jen řádky 71--77 z funkce \verb"\kdyzjsemjaslouzil_vysazej_sloku:nn".

\section{Závěr}
\label{sec:zaver}

Díky programovacímu jazyku expl3 mohou uživatelé \TeX u programovat způsobem, na který jsou zvyklí z moderních imperativních programovacích jazyků.

V článku jsem představil modul \pkg{l3seq} jazyka expl3, který poskytuje datovou strukturu seznamu. Možnosti modulu jsem demonstroval na sazbě textu české lidové písně \emph{Když jsem já sloužil}.
Dále jsem ukázal, že text písně je možné stručně a čitelně vysázet v také jazyce plain \TeX{}, pokud se programátor přizpůsobí omezenému počtu základních datových struktur v plain \TeX u.

\section*{Odkazy}
\vspace*{-0.1cm}
\printbibliography[heading=none]

\begin{summary}
The language of plain \TeX{} was developed for typesetting books and only became a Turing-complete programming language at the end of its development. Whereas writing and designing documents is straightforward in plain \TeX, programming is difficult due to a lack of basic data structures and the delayed macro expansion, which is different from modern imperative programming languages.

In the Lua\TeX{} engine, authors can also program in the imperative programming language of Lua. Although Lua does not share the limitations of plain \TeX, passing data between \TeX{} and Lua is not straightforward and important information such as token category codes are lost in transit.

The expl3 programming language combines the best of both worlds and allows authors to program in \TeX{} in a way that is similar to modern imperative programming languages.

In this article, I introduce the \pkg{l3seq} module of the expl3 language that provides the list data structure. Using \pkg{l3seq}, I typeset the lyrics of the Czech folksong \emph{Když jsem já sloužil}. I also compare the \pkg{l3seq} implementation with plain \TeX.

\keywords: \LaTeX3, expl3, \pkg{l3seq}, \pkg{xparse}, plain \TeX, Op\TeX, song lyrics
\end{summary}

\directlua{
  os.execute("cat example-01.tex example-02.tex example-03.tex > priklad-latex.tex")
  os.execute("cp example-optex.tex priklad-optex.tex")
  os.execute("cat kdyz-jsem-ja-slouzil-*.sty > kdyz-jsem-ja-slouzil.sty")
}
\end{document}